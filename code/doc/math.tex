\documentclass{article}
\usepackage{amsmath}

\begin{document}

We can include in-line math by surrounding it with the dollar-sign (\$).  For
example, commutative addition may be expressed as $a+b=b+a$. We can express
fractions using \texttt{frac}, as in $\frac{a}{b}$. Subscripts such as $a_b$
and superscripts like $a^b$ are possible; and subscript expressions $a_{n+1}$
or $b^{n-1}$.

Following is the equation for the sum of numbers from 1 to $n$:

\begin{equation}
  \displaystyle\sum_1^n i = \frac{n(n+1)}{2}.
\end{equation}

Following is the power rule for integration. If we place the equation in
escaped brackets instead of the \texttt{equation} environment, it does not
number the equation:

\[
  \int x^r dx = \frac{x^{r+1}}{r+1} + C.
\]

Power rule for differentiation, in-line: $\frac{d}{dx} x^r dx = rx^{r-1}$.
Also, there is a verbatim environment that we can use for code:

\begin{verbatim}
#include <iostream>

int sum(int n) {
  return n*(n+1)/2;
}

int main() {
  std::cout << sum(100) << std::endl;
}
\end{verbatim}

In addition, we can use the Greek alphabet: $\alpha$, $\beta$, $\theta$,
$\Sigma$, $\Pi$, etc. can easily be referred to by name. 

\end{document}
